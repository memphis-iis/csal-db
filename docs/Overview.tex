\documentclass[letterpaper,10pt]{article}

\usepackage{fullpage}

\usepackage{amsmath}
\usepackage{amssymb}
\usepackage{amsthm}
\usepackage{nameref}
\usepackage{url}

\usepackage[underline=true,rounded corners=false]{pgf-umlsd}

\title{CSAL Database Project Introduction}
\author{Craig Kelly}

% Macro for section num AND section name references
\newcommand{\fullxref}[1]{ \ref{#1} \nameref{#1} }

\begin{document}

% Force pdflatex to properly use letter as page size (instead
% of defaulting to A4)
\special{papersize=8.5in,11in}
\setlength{\pdfpageheight}{\paperheight}
\setlength{\pdfpagewidth}{\paperwidth}

\setlength{\parindent}{0pt}
\setlength{\parskip}{6pt}

\maketitle
\tableofcontents
\pagebreak

%%%%%%%%%%%%%%%%%%%%%%%%%%%%%%%%%%%%%%%%%%%%%%%%%%%%%%%%%%%%%%%%%%%%%%%%%%%%
%%%%%%%%%%%%%%%%%%%%%%%%%%%%%%%%%%%%%%%%%%%%%%%%%%%%%%%%%%%%%%%%%%%%%%%%%%%%

\section{Introduction}

This repository contains code for storing and displaying data stored as
part of the CSAL project.  The use cases and the code produced are described
below.  The short version is that the data is stored in a MongoDB instance,
there is a C\# library for accessing the database, and there is a Web API
wrapping the DLL.  In addition, the Web API server provides a very simple
ASP.NET MVC application for viewing the data.  Because of the dependence
of the Web API on the "core" library, the only deployment information is below
in \fullxref{subsec:deploy}

%%%%%%%%%%%%%%%%%%%%%%%%%%%%%%%%%%%%%%%%%%%%%%%%%%%%%%%%%%%%%%%%%%%%%%%%%%%%

\section{Use Cases}

\subsection{Direct Logging}
\label{use:directwrite}

Services running on the same server as the database (i.e. ACE) want the
ability to write JSON data log entries without POST'ing to an HTTP endpoint.
Note that while the architecture of the project allows this use case (and it
is currently used in production), a safe alternative would be to force all
applications to write via the ReST API (see \fullxref{use:restwrite}).

\subsection{Logging via ReST Endpoint}
\label{use:restwrite}

Applications may post a JSON record representing an ACE turn to a public
endpoint for persisting in the database. Note that this is the recommended
way to save data to the database (contrast with \fullxref{use:directwrite})

\subsection{Teacher Status Check}

Teachers need to be able to see students' progress in the lesson.  There
should be a way to see how the entire class is doing, how the class is doing
on a lesson, how a student is doing across the lessons, and how a specified
student is doing on a specific lesson.  Note that if this application grows,
this Use Case should be broken out into small chunks

\subsection{Data Administration}

There must be a way for administrators (not teachers) to initialize and edit
the database.  Specifically, class, lesson, and student data should be
configured for expected logging (via either use case
\fullxref{use:directwrite} or \fullxref{use:restwrite}).

%%%%%%%%%%%%%%%%%%%%%%%%%%%%%%%%%%%%%%%%%%%%%%%%%%%%%%%%%%%%%%%%%%%%%%%%%%%%

\section{Database}

The main documentation for the JSON logging record is available in a Google
doc.  Please see the documents ``CSAL Data'' at 
\url{https://docs.google.com/document/d/19nJZMReWpTat_tjNeOvA7oa8rDD0KmQ5oKaHnn2HmwE}
and ``AutoTutor Conversation Engine (ACE) Web API (CORS Version)'' at
\url{https://docs.google.com/document/d/1ZRlj7e5u4PQSlggCD--yZ5EsB2KCZzt7P8MEI6HB9So}

The various database entities are described below.  The JSON logging data
described above is stored in the entity described in \fullxref{db:turns}. You
may also see how the C\# classes for this data (both the JSON logging format
and the database entities below) are structured by looking in the CSALMongo
project or the CSALMongo.chm compiled documentation in this directory.


The server is autotutor.x-in-y.com and the MongoDB database should be named
csaldata.  There are four collections: classes, lessons, students, and
studentActions.

\subsection{Class}
\label{db:class}

There is one document in this collection per class.  It contains a list of the
students in the class and the lessons used. Please see \fullxref{db:turns} for
details on auto-creation and updating.

\subsection{Student}
\label{db:student}

There is one document in this collection per student. Please see
\fullxref{db:turns} for details on auto-creation and updating.

\subsection{Lesson}
\label{db:lesson}

There is one document in this collection per lesson. Please see
\fullxref{db:turns} for details on auto-creation and updating.

\subsection{Student Actions}
\label{db:turns}

There is one document in this collection per student per lesson. Any time a
JSON logging record is saved for a student in a lesson, it is appended to
the Turns list in the corresponding document in this collection. Although it
is preferred to have the class, lesson, and student documents matching this
information pre-populated, a minimal version of each entity will be created
if it is not present when the data is logged.  Auto-created entities will have
a property named AutoCreated set to true.

To help with queries, we also update the class, student, and lesson documents
when we receive turn data like so:

\begin{itemize}
    \item \fullxref{db:class} - update the fields Students and Lessons
    \item \fullxref{db:lesson} -  update the fields LastTurnTime, Students,
          AttemptTimes, StudentsAttempted, StudentsCompleted, and URLs
    \item \fullxref{db:student} -  update the fields LastTurnTime and TurnCount
\end{itemize}



%%%%%%%%%%%%%%%%%%%%%%%%%%%%%%%%%%%%%%%%%%%%%%%%%%%%%%%%%%%%%%%%%%%%%%%%%%%%

\section{CSALMongo DLL}

The ``base'' or ``core'' DLL contains the model classes for JSON logging
format, the model classes for the database, the actual database interface
class, and some supporting code.  The project is documented via XML documentation
which has been compiled into the CSALMongo.chm in this directory.

There is also a unit test project named CSALMongoUnitTest. It uses the Unit
Testing facilities availble with Visual Studio 2013. The tests are broken into
five categories of testing:

\begin{enumerate}
    \item Model Testing for methods added the model classes for
          parsing, information, etc.
    
    \item Database Operations Testing for actual database operations exposed
          by the main class
    
    \item Database Utility Testing for helper or utility methods exposed by
          the main database class
    
    \item Utility Testing for helper or utility functions \textbf{outside} the 
          main database class.
\end{enumerate}



%%%%%%%%%%%%%%%%%%%%%%%%%%%%%%%%%%%%%%%%%%%%%%%%%%%%%%%%%%%%%%%%%%%%%%%%%%%%

\section{CSALMongo Web API}
\label{sec:webapi}

\subsection{ReST API}

The ReST API is exposed via a .NET Web Api project (that also houses a GUI - 
see \fullxref{web:gui}).

Since this is a ReST API, there is a URL namespace complete with expected verbs
and payloads.  They are documented below.  It should be assumed that for local
workstation testing the url would begin with \url{http://localhost:62702}.  For
the production URL, the proper prefix would be \url{http://autotutor.x-in-y.com/csaldb}.

\begin{itemize}
    \item TODO
\end{itemize}

\subsection{User Interface}
\label{web:gui}
The User Interface is an ASP.NET MVC web application, served by the Home controller
class, rendered via Razor.  It uses jQuery and Bootstrap for UI automation
and styling.  Various jQuery UI plugins are also used (notably DataTables and
Sparklines).

The application maintains a URL namespace similar to the ReST API, but under
the Home directory.  As above, It should be assumed that for local
workstation testing the url would begin with \url{http://localhost:62702}.  For
the production URL, the proper prefix would be 
\url{http://autotutor.x-in-y.com/csaldb}.

\begin{itemize}
    \item TODO
\end{itemize}

\subsection{Authentication}

Authentication is handled via Google OAuth2.  Administrators are identified
by email address in the web.config file.  Teachers are given access to class
information if the address from their OAuth2 profile matches the teacher name
for the class.

\subsection{Logging}

This document generally refers to logging to identify ACE turn records
sent in JSON format and stored in the MongoDB collection studentActions.
However, the Web API and GUI must log records as well.  Currently this is
fairly simple; in addition to default IIS logging, the Elmah library is
used to log unhandled exceptions and any errors displayed via the custom
user error page.  The Elmah log is only stored in-memory (and so is transient)
and can be accessed at the main URL at
\url{http://autotutor-x-in-y.com/csaldb/elmah}.

\subsection{Deployment}
\label{subsec:deploy}

The entire application should be deployed via Visual Studio packaging and IIS
application import.  Please note that the application is already deployed to
its own App Pool on the production server, so a new deployment should be an
overwrite (not a new application).

Any deployment should also be accompanied by an annotated tag in the Git
repository.

%%%%%%%%%%%%%%%%%%%%%%%%%%%%%%%%%%%%%%%%%%%%%%%%%%%%%%%%%%%%%%%%%%%%%%%%%%%%

\section{Sequence Diagrams}

\subsection{Example of User Interaction}

\begin{sequencediagram}
    \newthread{web}{GUI User}
    \newinst[1]{api}{Web API}
    \newinst[1]{dll}{CSALMongo DLL}
    \newinst[1]{mongo}{MongoDB}
    
    \begin{call}{web}{View Lessons}{api}{render view}
        \begin{call}{api}{findLessons()}{dll}{return list}
            \begin{call}{dll}{find()}{mongo}{query response}
            \end{call}
        \end{call}
    \end{call}
\end{sequencediagram}

\begin{sequencediagram}
\end{sequencediagram}


%%%%%%%%%%%%%%%%%%%%%%%%%%%%%%%%%%%%%%%%%%%%%%%%%%%%%%%%%%%%%%%%%%%%%%%%%%%%
%%%%%%%%%%%%%%%%%%%%%%%%%%%%%%%%%%%%%%%%%%%%%%%%%%%%%%%%%%%%%%%%%%%%%%%%%%%%

\end{document}
